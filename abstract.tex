\noindent
Color Transparency (CT) is a prediction of QCD that at high momentum transfer
$Q ^2$, a system of quarks which would normally interact strongly with nuclear
matter could form a small color-neutral object whose compact transverse size
would be maintained for some distance, passing through the nucleus undisturbed.
A clear signature of CT would be a dramatic rise in nuclear transparency $T$
with increasing $Q^2$. The existence of CT would contradict traditional Glauber
multiple s cattering theory, which predicts constant $T$. CT is a prerequisite
to the validity of QCD factorization theorems, which provide access to the
generalized parton distributions that contain information about the transverse
and angular moment a carried by quarks in nucleons. The E12-06-107 experiment
at JLab measured $T$ in quasielastic electron-proton scattering with carbon-12
and liquid hydrogen targets, for $Q^2$ between 8.0 and
\SI{14.2}{\giga\electronvolt\squared}, a range over which models of CT
predicted that $T$ might differ appreciably from Glauber calculations.
Supported in part by US DOE grant DE-FG02-03ER41240.
