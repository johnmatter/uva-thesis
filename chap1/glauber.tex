\section{Glauber Multiple Scattering Theory}
This experiment takes the work of Pandharipande and
Pieper~\cite{Pandharipande_1992} as the null hypothesis against which the onset
of color transparency is to be tested.
This is the same model used in previous measurements of nuclear transparency
for quasielastic scattering. % TODO: cite previous measurements; garrow etc


The model starts with the assumption that the differences between cross
sections for free and in-medium nucleon-nucleon scattering arise primarily from
Pauli blocking of final states and effective mass corrections.
Pandharipande and Pieper find good agreement between experimental results and
their model's estimates of the imaginary part of the optical potential in
nuclear matter.


The dispersion relation $e(k,\rho)$ for nucleons in nuclear matter with density
$\rho$ and real part of the optical potential $U(k,\rho)$ is
\begin{equation}
    e(k,\rho) = \frac{\hbar^2 k^2}{2m} + U(k,\rho)
\end{equation}

The velocity of an in-medium nucleon with momentum $k$ differs from that of a
free nucleon.
The derivative of the dispersion relation gives this velocity and a
definition of an effective mass $m^*$
\begin{equation}
    \frac{1}{\hbar}\frac{de(k,\rho)}{dk}
        = \frac{\hbar^2 k}{m} + \frac{1}{\hbar}\frac{d}{dk}U(k,\rho)
        \equiv \frac{\hbar k}{m^*(k,\rho)}
\end{equation}

% With this velocity $v'$, transition matrix $t'$, and density of states $D'$,
% the in-medium nucleon-nucleon scattering cross section in a volume of size
% $L^3$ can be expressed as
% \begin{equation}
%     \frac{d\sigma}{d\Omega} = \frac{L^3}{v'} \frac{2\pi}{\hbar} |t'|^2 D'
% \end{equation}
% The equivalent expression holds with unprimed quantities for the vacuum cross
% section.
% Assuming the in-medium correlated two-nucleon wave function at small
% interparticle distances is the same as in the vacuum, the transition matrix
% $t'$ can be taken to be $t' \approx t$.
% Using the approximation from Ref~\cite{Krotscheck_1981}, the in-medium density
% of states is
% \begin{equation}
%     D' = D \frac{m^*\left( \sqrt{\frac{1}{2}(k_3^2+k_4^2)}, \rho \right)}{m}
% \end{equation}
% and the in-medium differential cross section is
% \begin{equation}
% \begin{aligned}
%     \frac{d \sigma^{\prime}}{d \Omega}=& \frac{v_{\mathrm{rel}}}{v_{\mathrm{rel}}^{\prime}} \frac{D_{f}^{\prime}}{D_{f}} \frac{d \sigma}{d \Omega} \\
%     =& \frac{\left|\mathbf{k}_{1}-\mathbf{k}_{2}\right|}{m}\left[\left|\frac{\mathbf{k}_{1}}{m^{*}\left(k_{1}, \rho\right)}-\frac{\mathbf{k}_{2}}{m^{*}\left(k_{2}, \rho\right)}\right|\right]^{-1} \\
%     & \times \frac{m^{*}\left[\sqrt{\left(k_{3}^{2}+k_{4}^{2}\right) / 2}, \rho\right]}{m} \frac{d \sigma}{d \Omega}
% \end{aligned}
% \end{equation}

The in-medium cross section for a proton with momentum $k$ scattering off a
nucleon $a=n,p$ is
\begin{equation}
    \widetilde{\sigma}_{pa}(k,\rho)=\frac{m^{*}(k, \rho)}{\hbar k \rho_{a} \tau_{a}(k)}
\end{equation}
where $\tau_a$ is the life time of the two-particle-one-hole state.
The model uses the
Urbana $\text{v}_{14}+\text{TNI}$ Hamiltonian~\cite{Lagaris_1981_331, Lagaris_1981_349}
and variational method~\cite{Wiringa_1988, Friedman_1981}
to calculate an optical potential $U$ for symmetric nuclear matter.


Predicts constant $T$.
