\section{Overview}
Color transparency (CT), a characteristic prediction of QCD, refers to the
reduction of initial and final state interactions between a hadron and the
nuclear medium in exclusive processes at large momentum transfer $Q^2$.
The concept was first proposed by Mueller and Brodsky in the context of
perturbative QCD, but was later shown to arise in nonperturbative models.
An analogue of CT can be seen in QED; a small $e^+e^-$ pair has a small cross
section determined by its dipole moment.


The three requirements for CT are the following:
\begin{itemize}
    \item Squeezing: the formation of a small configuration of quarks, sometimes
          referred to as a point-like configuration (PLC)
    \item This PLC is color-neutral outside its radius
    \item Freezing: the PLC maintains its small size over a distance comparable
          to or greater than the nuclear radius
\end{itemize}
There is theoretical support for selection of PLCs in exclusive processes.
Strength of interaction is proportional to transverse size of hadron interacting
with the nuclear medium.
Freezing time can be approximated and has been studied.
Moreover, CT is a necessary condition for the validity of QCD factorization
theorems.


Previous experiments looking for the onset of CT have been suggestive of the
meson electro/photoproduction
A(p,2p) at BNL
A(e,e'p) at SLAC and JLab


In quasielastic scattering experiments, a common observable is the nuclear
transparency $T=\sigma_A/A\sigma_0$, the ratio of the nuclear cross section per
nucleon to the cross section for a free nucleon.
Traditional Glauber multiple scattering theory predicts that $T$ is constant as
$Q^2$ increases.
The reduction of inital/final state interactions predicted by CT results in an
increase in nuclear transparency with $Q^2$.


Previous measurements of nuclear transparency in quasielastic electron
scattering experiments have been consistent with the Glauber prediction.
The goal of this experiment was to extend the range of $Q^2$ studied in the
search for the onset of CT.
