\section{Plane Wave Impulse Approximation (PWIA)}

Let $E_e$ and $E'_e$ be the incoming and outgoing energies of an electron
scattering elastically from a proton.
The cross section for this process, calculated for single photon exchange,
is given by the Rosenbluth formula
\begin{equation}
\frac{d\sigma}{d\Omega} = \left( \frac{d\sigma}{d\Omega} \right)_{Mott}
                          \frac{E'_e}{E_e}
                          \left(
                                \frac{G_E^2 + \tau G_M^2}{1+\tau} +
                                2 \tau G_M^2 \tan^2 \frac{\theta}{2}
                          \right)
\end{equation}
where
$G_E$ and $G_M$ are the electric and magnetic form factors,
$q^\mu=(\omega,\vec{q})$ is the 4-momentum transferred to the proton,
$Q^2=-q_\mu q^\mu$ is the momentum transfer squared,
$\tau=Q^2/4M^2$,
and $\left( \frac{d\sigma}{d\Omega} \right)_{Mott}$ is the cross section for
elastic scattering off a structureless point particle,

\begin{equation}
    \left( \frac{d\sigma}{d\Omega} \right)_{Mott} =
                    \frac{\alpha^2 \cos^2 \frac{\theta}{2}}
                         {4E_e^2 \sin^4 \frac{\theta}{2}}
\end{equation}

Because nucleons bound in nuclear targets are off-shell and the particles
involved can interact with the surrounding nuclear medium, the Rosenbluth model
is not valid in this case.

\begin{figure}[H]
    \centering
    \vspace{1cm}
        \begin{fmffile}{chap1/quasielastic_scattering}
            \setlength{\unitlength}{1cm}
            \begin{fmfgraph*}(8,5)
                \fmfleft{ie,oe}
                \fmfright{iA,oAm1,op}

                \fmflabel{$(E_e,\vec{p}_e)$}{ie}
                \fmflabel{$(E'_e,\vec{p'}_e)$}{oe}

                \fmflabel{$(E_A,\vec{p}_A)$}{iA}
                \fmflabel{$(E'_p,\vec{p'}_p)$}{op}
                \fmflabel{$(E_{A-1},\vec{p}_{A-1})$}{oAm1}

                \fmf{fermion}{ie,v1,oe}

                \fmf{photon,label=$q_\mu=(\nu,,\vec{q})$}{v1,v3}

                \fmf{dbl_plain_arrow}{iA,v2}
                \fmfblob{16}{v2}
                \fmf{fermion,label=$(E_s,,\vec{p}_p)$}{v2,v3}
                \fmf{dbl_plain_arrow}{v2,oAm1}
                \fmf{fermion}{v3,op}
            \end{fmfgraph*}
        \end{fmffile}
    \vspace{1cm}
    \caption{Feynman diagram for $A(e,e'p)$ scattering in the PWIA.}
    \label{fig:feynman_aeep}
\end{figure}


The plane wave impulse approximation (PWIA) incorporates these complications
with the following assumptions and approximations:
\begin{enumerate}
    \item Individual nucleons interact with the mean field generated by the rest
        of the nucleus, with no current exchanged between nucleons
    \item Free form factors can be used to describe bound nucleons
    \item The electron and proton's initial and final state wavefunctions are
        undistorted plane waves (meaning there are no ISI, FSI, or Coulomb
        distortions)
    \item The outgoing proton absorbed the entire momentum transfer
    \item Single photon exchange is sufficient to describe the interaction
\end{enumerate}

Incorporating off-shell effects, the differential cross section can be
factorized~\cite{Dieperink_1975, DeForest_1983, Frullani_1984}
% 1975 is the review paper of PWIA
% 1983 is the paper that kind of generalizes the 1976 He3 paper
% 1984 is a more recent review that I had trouble finding, found(!), lost, but can't find anymore
\begin{equation} \label{eqn:born_cross_section}
    \frac{d^6 \sigma}{dE'_{e} d\Omega'_{e} dE'_{p} d\Omega'_{p}} = p'_{p} E'_{p} \sigma_{ep} S(E_s, \vec{p}_p)
\end{equation}
where $\Omega'_{e}$ and $\Omega'_{p}$ are the solid angles of the outgoing particles,
$\sigma_{ep}$ is the off-shell $ep$ cross section,
and the spectral function $S(E_s, \vec{p}_p)$ represents the probability of
finding a proton in the nucleus with initial momentum $\vec{p}_p$ and
separation energy $E_s$.
The separation energy is the energy required to remove a proton from the nucleus
to infinity while leaving the recoil nucleus with zero kinetic energy
$T_{A-1}=0$.
In this work, the $ep$ cross section used is DeForest's
prescription~\cite{DeForest_1983} $\sigma^{cc}_1$,
which was initially calculated in Ref~\cite{Dieperink_1976} for quasielastic
scattering from ${}^3He$.
This cross section is set by imposing momentum and energy conservation at the
$\gamma p$ vertex.
% i.e. J dot q = rho nu
The normalization condition, given nuclear charge $Z$, for the spectral
function is
\begin{equation} \label{eqn:spectroscopic_sum_rule}
\int d^3p_p dE_s S(E_s, \vec{p}_p) = Z
\end{equation}

$E_s$ and $p_p$ can be estimated by measuring missing energy and momentum
\begin{equation}
    E_m = \omega - T_{p'} - T_{A-1}
\end{equation}
\begin{equation}
    \vec{p}_m = \vec{p}_{p'} - \vec{q}
\end{equation}
where the kinetic energy is $T=E-m$.
These measured quantities differ from $E_s$ and $p_p$ due to FSI, radiative
effects, and the spectrometers' finite resolutions.


The cross section in Equation~\ref{eqn:born_cross_section} assumes the recoil
nucleus $A-1$ remains in its ground state.
Coincidence data can include events where this is not the case, for
instance where the recoil nucleus is excited or one spectrometer arm is
triggered by a pion.
Data taken at Saclay~\cite{Mougey_1976} for quasielastic scattering from
${}^{12}C$
suggest that a cut on missing energy below $\SI{\sim100}{MeV}$ limits the rates
of the former.
Cutting on missing energy and using cuts on quantities from particle
identification (PID) detectors limits coincidence events other than $ep$.
