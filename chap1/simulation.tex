\section{Simulation}
The simulated yields in equation~\ref{eqn:transparency_definition} come from
Monte Carlo simulations of scattering processes, radiative effects,
and spectrometer performance.
These simulations are carried out by the FORTRAN program
\textit{SIMC}~\cite{simc_github, simc_wiki}.


\textit{SIMC} was initially written for the NE18 experiment at
SLAC~\cite{Makins_1994} and subsequently adapted for the HMS and SOS
spectrometers in JLab's Hall C.
Further development added support for more scattering processes, the pair of
HRS spectrometers in Hall A, and more recently the new SHMS spectrometer in
Hall C.


\textit{SIMC} generates events over a wide phase space, starting from beam and
target geometry.
Generating events over a region of phase space wider than the spectrometers'
acceptance allows simulation of events that will be thrown into the acceptance
window, for example because of multiple scattering or energy loss.
Events are then propagated through the spectrometers, accounting for final
state interactions, energy loss, multiple scattering, and spectrometer
acceptance and resolution.
Target variables are reconstructed from tracks fit at the focal plane.
Then a weight is calculated based on a model cross section for the initial
kinematics of each event.


Appendix~\ref{app:simc} contains descriptions of some of the models and
 parameters used in \textit{SIMC} simulations.
