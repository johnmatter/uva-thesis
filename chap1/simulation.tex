\section{Simulation}
The simulated yields are taken from Monte Carlo simulations of scattering
processes, radiative effects, and spectrometer performance.
These simulations are carried out by the FORTRAN program
\textit{SIMC}~\cite{simc_github, simc_wiki}.


\textit{SIMC} was initially written for the NE18 experiment at
SLAC~\cite{Makins_1994} and subsequently adapted for the HMS and SOS
spectrometers in JLab's Hall C.
Further development added support for more scattering processes, the pair of
HRS spectrometers in Hall A, and more recently the new SHMS spectrometer in
Hall C.


\textit{SIMC} generates events over a wide phase space, starting from beam and
target geometry.
Generating events over a region of phase space wider than the spectrometers'
acceptance allows simulation of events that will be thrown into the acceptance
window, for example because of multiple scattering or energy loss.
Events are then propagated through the spectrometers, accounting for final
state interactions, energy loss, multiple scattering, and spectrometer
acceptance and resolution.
Target variables are reconstructed from tracks fit at the focal plane.
Then weight is calculated based on a model cross section for the initial
kinematics of each event.


%------------------
% TODO: move most of this to chap 4 or an appendix?

% The general structure of SIMC's simulation for (quasi)elastic
% scattering is described below.
% \begin{enumerate}
%     \item Initialization
%     \begin{itemize}
%         \item Choose a reaction and final state
%         % \item Disable/enable implementation of (or correction for) raster, eloss ...
%     \end{itemize}

%     \item Event Generation
%     \begin{itemize}
%         \item Generate an event vertex based on target geometry,
%               beam width, beam raster, and beam energy
%         \item Generate $\theta_e$, $\phi_e$, $p_e$,
%                        $\theta_p$, $\phi_p$, $p_p$
%     \end{itemize}

%     \item Event Propagation
%     \begin{itemize}
%         \item Adjust event for radiative effects, Coulomb corrections, particle decays, etc.
%         \item Propagate particles through spectrometers using COSY models,
%               applying energy loss and multiple scattering in the target and
%               detectors.
%     \end{itemize}

%     \item Event Reconstruction
%     \begin{itemize}
%         \item Fit tracks in the focal plane
%         \item Reconstruct target variables $\delta$, $x'_{tar}$, $y'_{tar}$, $y_{tar}$
%     \end{itemize}

%     \item Normalization and Saving to Disk
%     \begin{itemize}
%         \item Calculate $E_m$ and $\vec{p}_m$ if simulating quasielastic scattering
%         \item Calculate per-event weight from spectral function, cross section,
%               radiative correction weight, and event generation weight.
%         \item Calculate normalization factor \textit{normfac} from luminosity,
%               phase space factors, and total number of events generated.
%         \item Save per-event physics quantities and weights in an Ntuple in a
%               PAW HBOOK.
%     \end{itemize}
% \end{enumerate}

% \textit{SIMC}'s output \texttt{.hbook} files can be converted to \texttt{.root}
% files for analysis alongside \textit{hcana} output using the utility
% \textit{h2root}.

% Good note explaining normfac
% https://hallaweb.jlab.org/12GeV/experiment/E12-07-108/Publications/Technical/Spectrometer/SIMC/simc_extra.pdf
% Gaskell/Arrington talk on simc
% https://hallaweb.jlab.org/collab/meeting/2009-winter/talks/Analysis%20Workshop%20--%20Dec%2014/simc_overview.pdf

\subsection{Spectral Functions}

The spectral functions used in SIMC are based on the independent particle shell
model (IPSM), which assumes nucleons occupy shells with quantum numbers $n$,
$l$, $j$, similar to the model of electron orbitals in atomic physics.
% preetty good slides
% http://indico.ictp.it/event/7641/session/21/contribution/46/material/0/0.pdf
In this model, the spectral function can be factored into a sum of per-shell
energy and momentum distributions
\begin{equation}
    S(E_m,\vec{p}_m) = \sum_i N_i \norm{\varphi_i(\vec{p})}^2 L_i(E_m)
\end{equation}
where $N_i$ is the occupation number of the $i$th shell,
$\varphi_i(\vec{p})$ is the bound state wavefunction,
and $L_i(E_m)$ is an energy profile.


% TODO: Give one of these Ls a tilde or prime to account for normalization?
The energy profile of each nuclear shell $i$ with binding energy $E_i$ is
given by a Lorentzian with finite width $\Gamma_i$ that accounts for the finite
lifetime of the one-hole state.
\begin{equation}
    L_i(E) = \frac{1}{\pi} \frac{\Gamma_i/2}{(E-E_i)^2 + \Gamma_i^2/4}
\end{equation}

The separation energy $E$ cannot be less than the minimum proton removal
energy $E_{min}=m_p + m_{A-1} - m_A$, so these profiles are cut off below
$E_min$ and normalized to ensure the spectroscopic sum rule,
Equation~\ref{eqn:spectroscopic_sum_rule}, is obeyed.
\begin{equation}
    L_i(E) =
    \begin{cases}
        L_i(E) / \int^{\infty}_{E_{min}} L_i(E) dE & \text{if $E \geq E_{min}$} \\
        0 & \text{if $E<E_{min}$}
    \end{cases}
\end{equation}


The wavefunctions $\varphi_i(\vec{p})$ are Fourier transforms of solutions
$\psi(\vec{r})$ to the Schroedinger equation with a potential given by the sum
of a
Woods-Saxon potential $-V_0 f(\vec{r})$,
Coulomb potential $V_C(r)$,
and spin-orbit coupling,
\begin{equation}
    V(\vec{r}) = - V_{0} f(r)
                 + V_{C}(r)
                 + V_{SO} \left(\frac{\hbar}{m_\pi c}\right)^{2}
                          \frac{2}{r} \frac{df}{dr}
                          \vec{l}\cdot\vec{s}
\end{equation}

The Woods-Saxon potential is characterized by
depth $V_0$,
radius $R_0=r_0(A-1)^{1/3}$, and
diffuseness $a$
parameters and a form given by a Fermi-Dirac distribution
\begin{equation}
    f(r) = \frac{1}{1+e^{\frac{r-R_0}{a}}}
\end{equation}
The Coulomb potential is that of a uniform sphere of radius
$R_c=r_c(A-1)^{1/3}$.


The model parameters were obtained from fits to the Saclay
measurements~\cite{Mougey_1976, Frullani_1984} of the ${}^{12}C$ spectral
functions.
The wavefunctions were obtained using the method described in
Ref~\cite{Giusti_1988, Giusti_1987, Blok_1991}.
% TODO: table of relevant parameters?


Short range nucleon-nucleon correlations push protons to higher missing energy
and momentum than accounted for by the IPSM.
Without correcting for this, simulated yields would be artificially large.
Assuming this leads to a uniform suppression of the spectral function below the
Fermi momentum, the spectral functions can be corrected by a constant factor,
assuming one consistently works in the same volume $V_m$ of $(E_m,\vec{p}_m)$ phase
space.
Given spectral functions $S_{SRC}$,that include the effects of short range
correlations, the correction is given by
\begin{equation}
    \frac{\int_{V_m} S_{IPSM}(E_m,\vec{p}_m) dE_m d^3p_m}
         {\int_{V_m} S_{SRC} (E_m,\vec{p}_m) dE_m d^3p_m}
\end{equation}

\subsection{Coulomb Corrections}
Coulomb distortions of the PWIA model arise from electromagnetic interactions
between the beam electron and target nucleus, modifying the momentum transfer
and incoming/outgoing momenta of the electron.


The energy required to bring an electron from infinity to a position $\vec{r}$
inside a nucleus with $Z-1$ protons is
\begin{equation}
    \Delta E(\vec{r})=f_{C}(|\vec{r}|)\left[\alpha \frac{(Z-1)}{R_{0}}\right]
\end{equation}
where
$\vec{r}=0$ is the center of the nucleus,
$\alpha$ is the fine structure constant,
and
$R_0=1.1 A^{1/3}+0.86 A^{-1/3}$ is the radius of the nucleus.


\textit{SIMC} uses the prescription described in Ref~\cite{Aste_2005}, which
takes this energy to be
\begin{equation}
    \Delta E = \frac{1.125(Z-1)\alpha}{R_0}
\end{equation}

Assuming the incoming electron is not deflected, its initial momentum at the
interaction vertex becomes
\begin{equation}
    (\vec{p}_e)_v = \vec{p}_e(1 + \Delta E / p_e)
\end{equation}
This value is used to calculate the momentum transfer
\begin{equation}
    \vec{q} = (\vec{p'}_e)_v - (\vec{p}_e)_v
\end{equation}
and opposite correction is applied to the outgoing electron momentum,
\begin{equation}
    \vec{p}_e = (\vec{p'}_e)(1 - \Delta E / (p'_e)_v)
\end{equation}

\subsection{Final State Interactions}

\subsection{Pauli Blocking}

\subsection{Radiative Corrections}

\subsection{Multiple Scattering}

\textit{SIMC} includes a list of the effective thickness in radiation lengths
$t_{eff}$ of every material a particle passes through.
As the simulated particle with momentum $p$ passes through each material, a
rescattering angle $\theta$ is calculated using the approximation of
Moli\`{e}re's multiple scattering theory given in Equation (6) of
Ref~\cite{Lynch_1991},
\begin{equation}
    \theta = \frac{E_s}{p \beta}
             \sqrt{t_{eff}}
             \left[1 + \epsilon \log_{10}{\left( \frac{t_{eff}}{\beta^2} \right)}\right]
\end{equation}
where $E_s=\SI{13.6}{\mega\electronvolt}$ and $\epsilon=0.088$ are parameters
taken from fits to multiple scattering measurements taken at
Fermilab~\cite{Shen_1979}.
These measurements were taken with incident pions, kaons, and protons at
momenta between 50 and \SI{200}{\giga\electronvolt} on hydrogen, beryllium,
carbon, aluminum, copper, tin, and lead targets.

\subsection{Energy Loss}
Energy loss in \si{\mega\electronvolt} is determined by the Bethe-Bloch formula,
the implementation of which can currently be found in \texttt{enerloss\_new.f}.
\begin{equation}
 E_{loss} =  K \frac{Z}{A} \frac{t}{\beta^2}
             \left[
                1.063
                + \log\left(\frac{m_e}{I^2}\right)
                + 2 \log(\gamma\beta)
                - \beta^2
                + \log\left(K\frac{Z}{A}\frac{t}{\beta^2}\right)
                - \delta
             \right]
\end{equation}
where
$K=\SI{0.1536e-03}{\centi\meter\squared\per\gram}$,
$t$ is the thickness of the material in \si{\gram\per\centi\meter\squared},
$Z$ and $A$ are the effective atomic number and weight of the material,
$I=16 Z^{0.9}) \si{\electronvolt}$ is the estimated ionization energy of the material,
and $\delta$ is a momentum-dependent density effect correction.
% Why 1.063? Trying to make sense of this and Leo secton 2.2.2

The density effect arises from the fact that a charged particle will polarize
the atoms in the material along its path.
Electrons in atoms far from this path will be shielded and contribute less to
the total energy loss.
The magnitude of this effect is greater at larger momenta.

\subsection{Weighting and Normalization}
\begin{equation}
    \text{normfac} = \frac{\mathcal{L} \Delta E_p \Delta \Omega_p \Delta E_e \Delta \Omega_e}
                          {N_{gen}}
\end{equation}



% TODO: describe FSI etc in some detail
