\section{Tracking Algorithm}
The \textit{hcana} tracking algorithm fits a trajectory to the position of hits
recorded by the drift chamber sense wires.
As discussed in Section~\ref{sec:dc_calib}, the transverse distance between a
sense wire and a particle passing by it can be precisely estimated by
converting a reference-time-subtracted TDC time to a drift distance.
Based on a single hit, it is impossible to tell whether the particle passed by
on the left or right side of the wire.
This ambiguity is resolved by looking at nearby pairs of planes that have
identical wire orientation and spacing, but with cells displaced by half a
cell.


In each drift chamber, hits that are close ``enough'' to each other are grouped
into \textit{space points}.
The hits in each space point are fit to create \textit{stubs}, halves of a
complete track.
Stubs in both chambers are paired to form candidate tracks that are saved for
further cleaning if the paired stubs are collinear ``enough.''


In the event that multiple track candidates are found for an event, two options
for selecting the best track (referred to as the \textit{golden track}) exist.
The simplest method selects the track with the lowest $\chi^2$ value.
The other method, referred to as \textit{pruning}, involves a series of quality
assurance tests that reject suboptimal tracks.
If none of the tracks pass all the tests, this method defaults to selecting the
track with the lowest $\chi^2$.
