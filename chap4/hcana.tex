\section{\textit{hcana}}
Hall C has developed a new analysis framework,
\textit{hcana}~\cite{hcana_github}, written in C++ to replace ENGINE, the old
analysis framework written in Fortran.
\textit{hcana} is an extension of the Hall A analyzer, Podd~\cite{podd_github},
a modular framework based on the CERN ROOT framework~\cite{cern_root}.
Event reconstruction or ``replay'' in \textit{hcana} proceeds in the following order:

% TODO: rewrite as pseudocode looping over events and detectors? could be messy. obviously don't want to get granular to the point of the hodoscope for-loop hell
% TODO: clarify that "detectors" aren't calculating anything; classes representing detectors calculate
\begin{enumerate}

    \item The analyzer unpacks hits stored as raw EVIO data and stores them in
        lists sorted by ROC, slot, and channel number.
        A text file containing a ``detector map'' associates these front-end
        electronics identifiers with a particular detector,
        plane/wire/PMT/etc., and type (ADC or TDC).

    \item The \texttt{Decode()} method converts these lists of raw hits into
        physically meaningful ADC quantities (pulse amplitude, pulse time,
        pulse integral, pulse pedestal) and TDC times for every detector.

    \item Tracking detectors run the \texttt{CoarseTrack()} method to generate
        a list of candidate tracks.

    \item Non-tracking detectors run the \texttt{CoarseProcess()} method,
        placing fiducial cuts on ADC and TDC values to select ``good'' hits.
        This step also performs any relevant calculation that does not require
        accurate tracking information, such as calculating the number of
        photoelectrons collected by each PMT in a Cherenkov detector.

    \item The analyzer then calculates precise tracking information for each
        event.
        All candidate tracks are transported from the focal plane to the target
        using the optics reconstruction matrix elements.

    \item The best track among all candidate, the ``golden track,'' is selected
        using one of three methods called Simple, Prune, and Scin.

    \item Non-tracking detectors run the \texttt{FineProcess()} method to
        calculate quantities that depend on accurate tracking information,
        such as the ``track-normalized energy'' deposited in the calorimeter,
        $E/p$.

    \item Finally, physics modules that calculate quantities such as invariant
        mass $W$, momentum transfer $Q^2$, and missing energy $E_m$ are run.

\end{enumerate}
% TODO: incldue description of EPICS decoding?


% Describing the software on this level might get messier than it's worth
% Each detector has a \texttt{Decode()} method that it inherits from the
% \texttt{THcDetector} class.
% Non-tracking detectors have \texttt{CoarseProcess()} and \texttt{FineProcess()}
% methods that calculate relevant quantities such as the total number of
% photolectrons for a Cherenkov detector.
% Similarly, tracking detectors have \texttt{CoarseTrack()} and
% \texttt{FineTrack()} methods.
% Every detector belongs to an instance of the \texttt{THcSpectrometer} class.


After this event loop is run for all events of interest in a raw CODA file,
TTrees containing reconstructed data are written to a ROOT file.
\textit{hcana} also allows the user to specify a set of one- and
two-dimensional histograms to be generated for each run.
Such standardized histograms are convenient for quality assurance and debugging
problems during data acquisition.
Replayed ROOT files can then be analyzed with further scripts written for the
purpose of detector calibration or physics analysis.


The sections that follow in this chapter will describe the details of each step
outlined in the list above.
