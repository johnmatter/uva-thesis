\section{Charge-normalized Yield}
Experimental yields for each kinematic settic were corrected using
kinematic-specific trigger efficiencies, PID efficiencies, tracking
efficiencies, and electronic livetimes.
Due to limited statistics, the same proton absorption correction calculated
using $Q^2=\SI{11.5}{\giga\electronvolt}$ was used for all kinematics.
The same LH2 boiling correction is used for all runs.

The carbon yields in both data and simulation were cut at
$E_{miss} < \SI{0.08}{\giga\electronvolt}$
$p_{miss} < \SI{0.3}{\giga\electronvolt}$.
For these cuts in carbon, the effect of nucleon-nucleon short-range
correlations was previously determined to shift the single particle strength
to higher $E_{miss}$ requiring a correction factor be applied to the data
1.11 $\pm$ 0.03\,\cite{ONeill_1995}.
These cuts and correction factor are the same ones that were used in previous
experiments measuring nuclear transparency for ${}^{12}C(e,e'p)$.
