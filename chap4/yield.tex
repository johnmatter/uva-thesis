\section{Charge-normalized Yield}
Experimental yields for each kinematic settic were corrected using
kinematic-specific trigger efficiencies, PID efficiencies, tracking
efficiencies, and electronic livetimes.
Due to limited statistics, the same proton absorption correction calculated
using $Q^2=\SI{11.5}{\giga\electronvolt}$ was used for all kinematics.
The same LH2 boiling correction is used for all runs.


The carbon yields in both data and simulation were integrated over
$E_{miss} < \SI{0.08}{\giga\electronvolt}$
$p_{miss} < \SI{0.3}{\giga\electronvolt}$.


Nucleon-nucleon short-range correlations shift the single particle strength
to higher $E_{miss}$ than predicted by the IPSM spectral functions.
To correct for this, a correction factor of 1.11 $\pm$ 0.03\,\cite{ONeill_1995}
is applied to the simulation yields.
This correction is the ratio of the values of the integral
$\int d^3p dE_m S(E_m,\vec{p}_m)$ 
calculated for IPSM and correlated~\cite{VanOrden_1980} spectral functions.
These cuts and correction factor are the same ones that were used in previous
experiments measuring nuclear transparency for ${}^{12}C(e,e'p)$.
