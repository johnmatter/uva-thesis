\section{Charge-normalized Yield}
Experimental yields $Y$ for each kinematic setting were corrected using
kinematic-specific efficiencies $\epsilon$
(PID, tracking, hodoscope), and
total electronic livetimes $TLT$.
Due to limited statistics, the same proton absorption correction $A$,
calculated using $Q^2=\SI{11.5}{\giga\electronvolt}$ data,
was used for all kinematics.
The same LH2 boiling correction for current $I$ in \si{\micro\ampere},
$f_{boil}(I)=1-0.000385I$
was used for all hydrogen runs.

The corrected experimental yield for each kinematic setting was obtained using
\begin{equation}
Y_{exp} = \frac{N_{coin}}{Q}
          \frac{f_{boil}(I)}{(1-A) * TLT}
          \left[
              \epsilon_{track}
              \epsilon_{cer}
              \epsilon_{cal}
              \epsilon_{hodo}
          \right]_{HMS}^{-1}
          \left[
              \epsilon_{track}
              \epsilon_{cer}
              \epsilon_{hodo}
          \right]_{SHMS}^{-1}
\end{equation}


The carbon yields in both data and simulation were integrated over
the the region
$E_{miss} < \SI{0.08}{\giga\electronvolt}$
and
$\norm{\vec{p}_{miss}} < \SI{0.3}{\giga\electronvolt}$; this cut removes
any accidental coincidences from pions in the SHMS.
Nucleon-nucleon short-range correlations shift the single particle strength
to higher $p_{miss}$ than predicted by the IPSM spectral functions.
To correct for this, a correction factor of 1.11 $\pm$ 0.03\,\cite{ONeill_1995}
is applied to the simulation yields.
This correction is the ratio of the values of the integral
$\int d^3p dE_m S(E_m,\vec{p}_m)$
calculated for IPSM and correlated~\cite{VanOrden_1980} spectral functions.
These cuts and correction factor were the same as the ones used in
previous experiments measuring nuclear transparency for ${}^{12}C(e,e'p)$.
This choice was made in the interest of consistency and ease of comparing these
new results with the prior data.



