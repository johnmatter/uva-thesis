\section{Defining Color Transparency}
The phenomenon known as color transparency (CT) was first proposed by
Mueller~\cite{Mueller_1982} and Brodsky~\cite{Brodsky_1982} in 1982.
It is a distinctive feature of QCD's quark degrees of freedom, not arising in a
purely hadronic model.
CT refers to vanishing initial and final state interactions (ISI and FSI)
between hadrons and the surrounding nuclear medium in exclusive processes
at large momentum transfer $Q^2$.
This is in contrast to conventional Glauber theory which assumes strong ISI/FSI
and rescattering.


A QED phenomenon analogous to CT can be seen in the Chudakov effect.
Consider experiments measuring the decay $\pi^0 \rightarrow e^+ e^- \gamma$
in photographic emlusions~\cite{Perkins_1955, Fowler_1955, Wolter_1956,
Iwadare_1958, Varfolomeev_1959, Zielinski_1985}.
As a electron-positron pair traveled through an emulsion,
ionization density increases with distance from the decay vertex,
consistent with suppressed interaction between a
small, slowly growing electric dipole and the surrounding medium.
A small $q\bar{q}$ or $qqq$ system or ``point-like configuration'' (PLC) is the
QCD analogue of the QED dipole\footnote{Incidentally, Bjorken used
the reverse of this analogy in 1976 to illustrate why a small
$q\bar{q}$ system shouldn't create jets in hadronic final
states created in electron-positron collisions~\cite{Bjorken_1976}.}.


The existence of CT requires the following criteria:
\begin{itemize}
    \item Scattering takes place by preferentially selecting hadrons with
          transverse size much smaller than the hadron's ``free'' radius.
    \item This small object is color neutral outside its radius, so that it
          does not radiate gluons, thus having reduced interactions with the
          nuclear medium.
    \item This compact size is maintained for a distance comparable to the size
          of the nucleus.
\end{itemize}


The simplest model describing CT is the two gluon exchange model developed by
Low and Nussinov~\cite{Low_1975, Nussinov_1975, Nussinov_1976}.
Expand on this?


\subsection{Squeezing}
The first criterion can be thought of as ``squeezing'' a quark system into a
PLC with tranverse size smaller than the radius of the hadron detected in
the final state.


% insert diagram for below?


An intuitive argument from Frankfurt, Miller, and
Strikman~\cite{Frankfurt_1992} is suggestive of the possibility of forming a
PLC in quasielastic electron scattering from nuclei.
Suppose a quark in the nucleus, after absorbing a virtual photon, is off-shell
by $\Delta E = Q$.
By the uncertainty principle, its lifetime should be $\tau=1/Q$.
It will decay by emitting a gluon which, if the final state is to include a
proton, must be absorbed by nearby quarks in a radius
$r \approx c \tau \sim 1/Q$.
Thus, for large momentum transfers, the quark system formed in the scattering
process should be quite small.


Importantly for CT, such a PLC will not interact with the nuclear medium if it
remains small; color-singlet point particles do not radiate gluons, and cannot
interact via gluon exchange~\cite{Gunion_1977}.


In 1980, Brodsky and Lepage~\cite{Brodsky_1980, Lepage_1980} showed, using
perturbative QCD (pQCD), that the ``squeezing'' criterion is satisfied for
exclusive processes at large $Q^2$.
In the years following, Isgur and Smith~\cite{Isgur_1984, Isgur_1988,
Isgur_1989} cautioned against the use of pQCD to study exlusive processes,
citing experimental evidence of significant soft contributions to pion and
nucleon form factors.
Experimental support for the dominance of PLCs in these processes will be
discussed in Section~\ref{sec:ct_intermediate_energies}.


\subsection{Freezing and Expansion}
Suppose a PLC is created in the interior of a nucleus and, in its rest frame,
expands to a configuration with normal size over a time $\tau_0$.
Taking time dilation into account, it expands in a time $\tau=\tau_0E/m$ in the
rest frame of the nucleus over a distance called the coherence length $l_c$.
For large enough energy $E$, $l_c$ is larger than the nuclear diameter and the
PLC can be described as ``frozen'' in its small transverse size as it escapes
the nucleus.
High energy processes where this is indeed the case will be discussed in
Section~\ref{sec:ct_high_energies}.


At intermediate energies however, one must take into account the expansion of
the PLC.
Using the uncertainty principle, the decoherence time can be
estimated, as in Ref~\cite{Farrar_1988}, for an intermediate PLC state with mass
$m_{inter}$ and ``normal'' mass $M_h$:
\begin{align}
    \Delta E &= \sqrt{p_h^2 + m_{inter}^2} - \sqrt{p_h^2 + M_h^2} \\
             &= p_h \left( \sqrt{1+\frac{m_{inter}^2}{p^2}} -
                           \sqrt{1+\frac{M_h^2}{p^2}} \right) \\
             &\approx p \left( \frac{m_{inter}^2}{2p_h} - \frac{M_h^2}{2p_h} \right) \\
             &= \frac{\Delta M_h^2}{2p_h}
\end{align}
where $\Delta M_h^2 = m_{inter}^2 - M_h^2$.
Then, in natural units, $\Delta E \Delta t = 1$ implies that the coherence
length is
\begin{equation}
    l_c = \frac{2p_h}{\Delta M_h^2}.
\end{equation}
The freezing approximation is valid if $l_c \gg R_A$ where $R_A$ is the radius of
the relevant nucleus.


Ref~\cite{Farrar_1988} also presents an estimate of the effective PLC-nucleon
cross section as a function of propagation distance $z$.
The model assumes that the effective cross section is scaled by the transverse
size of the quark system $x_t$ relative to the average size of the hadron
$\langle x_t \rangle$.
That is
$\sigma^{eff}_{hN} = \left[ x^2_t(z) / \langle x_t \rangle^2 \right] \sigma^{tot}_{hN}$


Let $n$ be the number of partons in the quark system,
$\langle k_t \rangle$ the average transverse momentum of a parton in the
hadron, and $t=-Q^2$ the momentum transfer squared.
Then the transverse area occupied by the quark system is
$\sigma^{tot}_{hN}(n^2 \langle k_t \rangle^2 / t)$ at the point of interaction.
The system expands over the coherence length $l_c$ to its normal hadronic size.


% TODO: change tau to another letter. Confusing because I also use it for lifetime.
% TODO: expand on the models?
% TODO: simplify and just use quantum diffusion?
The effective cross section is then
\begin{equation}
    \sigma_{hN}^{eff} = \sigma_{hN}^{tot}
    \left(
        \left\{\left(\frac{z}{l_c}\right)^{\tau} +
               \frac{\left\langle n^{2} k_{t}^{2}\right\rangle}{t} \left[1-\left(\frac{z}{l_c}\right)^{\tau}\right]
        \right\}
        \theta\left(l_c-z\right) +
        \theta\left(z-l_c\right)
    \right)
\end{equation}
The parameter $\tau$ distinguishes three models:
non-perturbative QCD ($\tau=0$; no reduction in cross section),
pQCD ($\tau=1$; $x_t$ grows like $\sqrt{z}$), and
a naive parton model ($\tau=2$; $x_t$ grows like $z$).


Another approach~\cite{Jennings_1990, Jennings_1991, Jennings_1992}
expands the PLC wave function in terms of hadronic eigenstates
$\ket{\psi_i}$ of the Hamiltonian.
Let $P$ be the PLC's mmoentum and assume each eigenstate satisfies
$E_i \gg m_i$.
Then
\begin{align}
    \ket{\psi_{PLC}(t)} &= \sum_{i=1}^{\infty} a_i e^{-iE_it}\ket{\psi_i} \\
                        &= e^{-iE_1t} \sum_{i=1}^{\infty} e^{-i\frac{(m_i^2-m_1^2)t}{2P}} \ket{\psi_i}
\end{align}

This suggests that the loss of coherence is due to the relative phase between
hadronic components.
In other words, the coherence length is the length at which coherence between
the lowest and first excited states is lost.


\subsection{subsection title (other connections?)}
CT was originally discussed in the context of pQCD, but has been shown to be
a general feature of other non-perturbative approaches~\cite{Frankfurt_1992}.


The existence of CT is a prerequisite for the validity of QCD factorization
theorems~\cite{Brodsky_1994, Collins_1997, Frankfurt_1999, Diehl_1998,
Strikman_2000} which provide access to the Generalized Parton Distributions
(GPDs) that currently provide the most complete picture of the internal
quark-gluon structure of various hadrons~\cite{Ji_1997_Jan, Ji_1997_Jun,
Radyushkin_1996, Radyushkin_1997}.
These theorems assume, at sufficiently large $Q^2$, that deeply inelastic
exclusive processes' amplitudes are separable into two parts: a hard scattering
at the parton level, and a soft part characterized by GPDs.


% insert that nice Feynman diagram


To illustrate the connection between CT and factorization, consider
meson electroproduction.
In the Breit frame, the virtual photon and baryon are both initially at rest.
After the meson absorbs the photon, the meson and baryon both move off in
opposite directions without any further gluon exchange between the two,
provided the meson maintains a small transverse size~\cite{Strikman_2000}.


Bjorken scaling?
% Section 4 of this long review
Relevant longitudinal distance is $y\sim 1 / 2 M_n x$~\cite{Frankfurt_1998}.
For small $x$, this is larger than the size of the nucleus.


A more in-depth discussion of these considerations can be found in recent
reviews by Dutta, Hafidi, and Strikman~\cite{Dutta_2013,Dutta_2012}.
