\section{Color Transparency at High Energies}
\label{sec:ct_high_energies}
\subsection{Pion dissociation into two jets}
Pion beam at fnal etc etc

One can treat the PLC as a ``frozen'' $q\bar{q}$ dipole with transverse 
size $d$~\cite{Blattel_1993, Frankfurt_1993}.
Then in the leading log approxiamtion, the dipole-nucleon cross section is 
given by~\cite{Frankfurt_2000, Frankfurt_2002}

\begin{equation}
    \sigma_{q\bar{q} N}^{inel}(d,x) = \frac{\pi^{2}}{3} \alpha_{s}
    \left( Q_{eff}^2 \right) d^2
    \left[
           x G_{N} \left( x, Q_{eff}^2 \right) +
           \frac{2}{3} x S_{N} \left( x, Q_{eff}^2 \right)
    \right].
\end{equation}

Here $Q_{eff}^2=\lambda/d^2$, $x=Q_{eff}^2/s$, $s$ is the invariant energy of
the dipole-nucleon system, and $S$ and $G$ are the sea quark and gluon
distributions making up the dipole.
The parameter $\lambda$ takes a values between 4 and 10, and was estimated by
matching this model with the leading log description of
$\sigma_L(x,Q^2)$~\cite{Frankfurt_1996}.

\subsection{$J/\psi$ photoproduction}

\subsection{Vector meson production}
